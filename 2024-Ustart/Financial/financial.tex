\section{財務計畫}

% TODO: 先找成本

\subsection{成本}

團隊主要的產品已軟體系統為主,因此成本主要集中在人力與設備上。人力成本主要是開發人員的薪資,設備成本主要是伺服器的租用費用。
在初期,團隊多以開發系統及收集實際使用者的回饋為主,因此行銷成本相對較低。

\begin{table}[H]     
  \caption{普羅程式-開發成本}
  \centering
  \begin{tabular}{|c|c|c|c|}
    \hline
    \thead{會計項目} & \thead{113年度} & \thead{114年度} & \thead{115年度} \\ 
    \hline
    薪資支出 & (240,000) & (480,000) & (960,000)  \\ 
    \hline
    伺服器租金費用 & (25000) & (50,000) & (150,000) \\ 
    \hline
    網域費用 & (1,000) & (1,000) & (1,000) \\
    \hline
    設備費 & (90,000) & (0) & (0) \\
    \hline
    視覺設計費 & (34,000) & (50,000) & (50,000) \\
  \end{tabular}
\end{table}

% 全民健康保險補充保費 
% 印刷費 20000
% 諮詢費 60000
% 雜支 15000

\begin{table}[H]     
  \caption{普羅程式-營運成本}
  \centering
  \begin{tabular}{|c|c|c|c|}
    \hline
    \thead{會計項目} & \thead{113年度} & \thead{114年度} & \thead{115年度} \\ 
    \hline
    薪資支出 & (90,744) & (120,000) & (240,000)  \\ 
    \hline
    設立公司費用 & (0) & (0) & (15,000) \\
    \hline
    辦公室租金 & (0) & (0) & (84,000) \\ 
    \hline
    美宣費用 & (0) & (25,000) & (50,000) \\
    \hline
  \end{tabular}
\end{table}

% TODO: 預期損益表

全台國高中學生約莫 100 萬人,若每人每年收費 300 元,第一年預估 0.1 \% 的市占
300 元 / 年 / 人 * 1000 人 = 30 萬元 / 年

第二年預估 1 \% 的市占,第三年預估 8 \% 的市占

% 2022的 ustart
% 營業收入 1,200,000 3,150,000 7,300,000
% 營業成本 (420,000) (660,000) (1,450,000)
% 營業毛利 780,000 2,490,000 5,850,000
% 營業費用
% //薪資支出 (624,000) (1,679,040) (2,060,640)
% //伺服器租金費用 (5,000) (50,000) (300,000)
% 租金 0 0 (700,000)
% 網域費用 (1,200) (1,000) (1,000)
% 印刷費 (5,000) (6,000) (6,000)
% 美宣費 (130,000) (150,000) (150,000)
% 廣告費 (20,000) (80,000) (180,000)
% 行銷宣傳顧問費 (80,000) (80,000) (80,000)
% 保險費 (118,560) (319,018) (391,522)
% 交際費 (5,000) (5,000) (5,000)
% 伙食費 (57,600) (115,200) (115,200)
% 職工福利 (8,800) (264,000) (324,000)
% 場地租用費 (20,000) (40,000) (40,000)
% 雜支 (15,000) (20,000) (20,000)
% 折舊 (24,000) (24,000) (224,000)