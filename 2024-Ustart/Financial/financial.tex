\section{財務計畫}

% TODO: 先找成本

\subsection{成本分析}

團隊主要的產品以軟體系統為主,主要聚焦於人力和設備成本。人力成本包括開發人員薪資,是我們主要的支出。設備成本則是伺服器租用費用。初期,我們專注於系統開發和收集使用者回饋,因此行銷成本較低。網域費用為簽訂10年約期,每年1000元。這樣的配置旨在確保資源有效利用,支持產品開發和市場進入。

\begin{table}[H]     
  \caption{普羅程式-開發成本}
  \centering
  \begin{tabular}{|c|c|c|c|}
    \hline
    \thead{會計項目} & \thead{113年度} & \thead{114年度} & \thead{115年度} \\ 
    \hline
    薪資支出 & (240,000) & (480,000) & (960,000)  \\ 
    \hline
    伺服器租金費用 & (25,000) & (150,000) & (350,000) \\ 
    \hline
    網域費用 & (1,000) & (1,000) & (1,000) \\
    \hline
    設備費 & (90,000) & (40,000) & (150,000) \\
    \hline
    視覺設計費 & (34,000) & (100,000) & (200,000) \\
    \hline
    場地租用費 & (10,000) & (30,000) & (100,000) \\
    \hhline{|=|=|=|=|}
    小計 & (400,000) & (801,000) & (1,761,000) \\
    \hline
  \end{tabular}
\end{table}

\begin{table}[H]     
  \caption{普羅程式-營運成本}
  \centering
  \begin{tabular}{|c|c|c|c|}
    \hline
    \thead{會計項目} & \thead{113年度} & \thead{114年度} & \thead{115年度} \\ 
    \hline
    薪資支出 & (90,744) & (120,000) & (240,000)  \\ 
    \hline
    設立公司費用 & (15,000) & (0) & (0) \\
    \hline
    辦公室租金 & (12,000) & (24,000) & (84,000) \\ 
    \hline
    美宣費用 & (3,000) & (25,000) & (50,000) \\
    \hline
    全民健康保險補充保費 & (6,978) & (12,600) & (25,200) \\
    \hline
    印刷費 & (5,000) & (20,000) & (40,000) \\
    \hline
    諮詢費 & (60,000) & (60,000) & (80,000) \\
    \hline
    旅費 & (15,000) & (20,000) & (40,000) \\
    \hline
    雜支 & (15,000) & (30,000) & (50,000) \\
    \hhline{|=|=|=|=|}
    小計 & (222,722) & (311,600) & (609,200) \\
    \hline
  \end{tabular}
\end{table}

\subsection{預期收入}

根據我們的市場分析,台灣國高中學生總數約為100萬人。我們計畫以B2B形式與學校合作,為每位學生提供我們的ProgLearn系統服務,收費標準為每位學生300元(教師帳號免費)。此外,ProgLearn系統將直接架設於學校的伺服器上,我們將提供系統維護等相關服務,以確保系統的穩定運行。

\begin{enumerate}
  \item 
  第一年:預計達到0.05\%的市場占有率,即吸引約500名學生使用我們的服務,年收入為15萬元。
  \item 
  第二年:預計市場占有率提升至0.5\%,相當於吸引約5000名學生,年收入為150萬元。
  \item 
  第三年:預計市場占有率將達到1.5\%,即約有1.5萬名學生使用我們的服務,年收入為300萬元。
\end{enumerate}

從第二年開始,我們將擴展至B2C市場,推出訂閱制服務,直接向教師收費。每個教師帳號的年費為3000元,這項服務將由我們自建的伺服器提供,確保教師能夠通過網頁服務輕鬆使用ProgLearn。每位訂閱的教師可免費提供給30位學生使用帳號(為期一年),如果需要超過這數量的學生帳號,將收取每位學生100元的年費,並按剩餘訂閱期按日計費。預計透過這種模式,第二年的收入將達到9萬元,第三年則能達到30萬元。

\subsection{預期損益表}

\begin{table}[H]     
  \caption{普羅程式-預期損益表}
  \centering
  \begin{tabular}{|c|c|c|c|}
    \hline
    \thead{會計項目} & \thead{113年度} & \thead{114年度} & \thead{115年度} \\ 
    \hline
    營業收入 & 150,000 & 1,590,000 & 4,300,000  \\ 
    \hline
    營業成本 & (622,722) & (1,112,600) & (2,370,200) \\
    \hline
    營業毛利 & (472,722) & 477,400 & 1,929,800 \\
    \hline
    業外收入 & 450,000\tablefootnote{U-start 計畫補助 350,000 元,Marker 計畫補助 100,000 元} & 1,000,000 & 0 \\
    \hhline{|=|=|=|=|}
    稅前盈餘 & (22,722) & 1,477,400 & 1,929,800 \\
    \hline
  \end{tabular}
\end{table}