\section{創業機會與構想}
\subsection{過去的創業學習經驗}

普羅程式於2021年成立,由三位海大資工系學生組成,成員中有具備全端工程師專業背景與發表AI相關論文於IEEE研討會的成員,並由海大資工的馬尚彬教授、魚樂天地鄉鎮應援團的何立德執行長分別擔任技術顧問和商業顧問。

團隊成員皆有豐富的程式設計、軟體開發、計畫執行、數位行銷等經驗,
包括參與112年學年度大專校院創業實戰模擬學習平臺之第二梯次\footnote{大專校院創業實戰模擬學習平臺(普羅程式):https://ssp.moe.gov.tw/cases/854}(圖\ref{fig:experience-1})
、提案於flyingV群眾募資平台\footnote{flyingV群眾募資平台(普羅Python入門課程):https://www.flyingv.cc/projects/29572}(圖\ref{fig:experience-2}、附件\ref{fig:Appendix-fundraising})
,並於海大教學中心參與多次創業研習(附件\ref{fig:Appendix-Training})。

\begin{figure}[H]
  \centering
  \begin{subfigure}{0.45\linewidth}
    \centering
    %   \href{https://raw.githubusercontent.com/programingtw/proglearn-plan/main/img/interactiveMeterial.png}{ 
    \includegraphics[width=0.8\textwidth]{images/maker.png}
    %   }
    \caption{大專校院創業實戰模擬學習平臺 - 提案封面}
    \label{fig:experience-1}
  \end{subfigure}
    \begin{subfigure}{0.45\linewidth}
    \centering
    %   \href{https://raw.githubusercontent.com/programingtw/proglearn-plan/main/img/interactiveMeterial2.png}{ 
    \includegraphics[width=0.8\textwidth]{images/flyingv.jpg}
    %   }
    \caption{flyingV群眾募資 - 提案封面}
    \label{fig:experience-2}
  \end{subfigure}
  \caption{創業經歷}
  \label{fig:Experience}
\end{figure}

此外,我們曾在全國性比賽中獲得多項獎項,包括2023資訊智慧創新跨域專題競賽特優獎(圖\ref{fig:Awards-1})並登上年代新聞(圖\ref{fig:News})、2021年潮創客大賽優選獎(圖\ref{fig:Awards-2})、2021武漢金銀湖盃第七屆海峽兩岸青年創新創業大賽入選台灣賽區菁英賽決賽、2021師大Startup競技場決賽(附件\ref{fig:Appendix-Competition})。

\begin{figure}[H]
  \centering
  \begin{subfigure}{0.32\linewidth}
    \centering
    %   \href{https://raw.githubusercontent.com/programingtw/proglearn-plan/main/img/interactiveMeterial.png}{ 
    \includegraphics[width=0.8\textwidth]{images/年代新聞.jpg}
    %   }
    \caption{年代新聞}
    \label{fig:News}
  \end{subfigure}
    \begin{subfigure}{0.31\linewidth}
    \centering
    %   \href{https://raw.githubusercontent.com/programingtw/proglearn-plan/main/img/interactiveMeterial2.png}{ 
    \includegraphics[width=0.45\textwidth]{images/ProgLearn.jpg}
    %   }
    \caption{資訊智慧創新跨域專題競賽}
    \label{fig:Awards-1}
  \end{subfigure}
  \begin{subfigure}{0.31\linewidth}
    \centering
      %   \href{https://raw.githubusercontent.com/programingtw/proglearn-plan/main/img/interactiveMeterial2.png}{ 
    \includegraphics[width=0.45\textwidth]{images/創博匯.jpg}
      %   }
    \caption{潮創客大賽}
    \label{fig:Awards-2}
  \end{subfigure}
  \caption{團隊榮譽}
\end{figure}

\subsection{創業機會}
在台灣,資訊科技領域備受關注。根據111年經濟部智慧學習產業產值調查報告\cite{ref:111產業產值調查報告}中,智慧學習軟體系統的產值為378.8億元,且每年有數百萬名國高中學生參與程式課程\cite{ref:學生數量},形成龐大的市場需求。軟體系統服務在2021年因疫情影響有顯著提升,成長了79.6\%,顯示軟體系統和線上教學是未來的趨勢。
% 在台灣,資訊科技領域備受關注。根據110年經濟部智慧學習產業產值調查報告\cite{ref:110產業產值調查報告}中,智慧學習軟體系統的產值為316.3億元,且每年有一百萬名國高中學生參與程式課程\cite{ref:學生數量},形成龐大的市場需求。軟體系統服務在2021年因疫情影響有顯著提升,成長了79.6\%,顯示軟體系統和線上教學是未來的趨勢。

此外,以視訊為主的教材與虛擬教學也成為培訓的主流\cite{ref:企業培訓},MOOCs平台如Udacity、Coursera、Intrepid等,協助Google、Microsoft、AT\&T等大型企業的培訓需求。而在台灣,也有許多教育機構如台灣大學、清華大學、交通大學等,提供具學分、證照、微學位等的線上課程,並有許多學生參與。

\subsection{創業構想}

在面對數位化學習的潮流下,學校、企業、個體教師、補教業者、程式才藝班等族群或機構紛紛加入了線上教學的行列,但是仍然存在著許多挑戰和需求\cite{ref:111產業產值調查報告}\cite{ref:110產業產值調查報告}。例如,數位轉型所具備的技術門檻、教育資源的整合和優化、個性化學習的需求、即時互動的困難與教師教學負擔的增加\cite{ref:老師的困難}等。因此,我們認為在這樣的市場環境下,有機會為客戶提供技術支援或教學工具,以解決這些問題。

普羅程式作為教學用軟體的服務供應商,將致力於協助教師開發程式教學工具,以解決教育者在數位化學習的潮流下所遇到的問題。
除了為教師與學校提供專案設計的服務外,我們將提供整合性的proglearn程式教學系統作為我們的主要產品,可適用於多種場景,如線上與線下課程、非同步與同步課程中。

\subsection{實施方式、時程規劃及預期成效}
\label{sec:plan}

本計畫將分為自辦、種子輔導、專案執行三個階段進行:

\begin{figure}[htbp]
  \centering
  \begin{tikzpicture}[node distance=2cm,>=latex']

      % Define block styles
      \tikzstyle{block} = [rectangle, draw, fill=blue!20, 
          text width=6em, text centered, rounded corners, minimum height=3em]
      \tikzstyle{line} = [draw, -latex']

      % Place nodes
      \node [block] (self_init) {自辦階段};
      \node [block, below=1.5cm of self_init] (seed_init) {種子輔導階段};
      \node [block, below=1.5cm of seed_init] (project_exec) {專案執行階段};

      % Draw edges
      \path [line] (self_init) -- (seed_init);
      \path [line] (seed_init) -- (project_exec);

  \end{tikzpicture}
  \caption{計畫實施流程圖}
  \label{fig:flowchart}
\end{figure}

\begin{enumerate}
  \setlength{\parindent}{2em}
  \item 自辦階段(2024年5月至2024年12月)
  \par 在自辦階段,我們正面臨著起步階段的挑戰,其中包括尚未建立起與學校之間的信任與合作關係,以及我們的系統缺乏實際的使用案例。為了應對這些挑戰,我們計劃利用proglearn程式教學系統,以我們的團隊成員作為第一批教師,舉辦各種活動以接觸基層的國高中教師與學生,例如小型營隊、課程、研討會或線上課程等。
  \par 這些活動將由開發人員進行教學,這樣我們便能夠即時修正系統問題,並且得到第一線教學的反饋。實施的具體方式是由學校提供場地和學生,而我們則提供課程內容、教師和教材,以建立與學校的初步合作關係。
  \par 這種方式對學校的成本較低,且風險相對較小,因此非常適合作為推廣的初期階段。透過這樣的合作模式,我們有望減少與學校合作過程中可能出現的摩擦,同時為我們的系統提供實際的應用場景,進一步提升市場接受度和使用者體驗。
  \item 種子輔導階段(2025年1月至2025年12月)
  \par 在當前階段,我們已經與少數學校建立了合作關係。接下來的重點是培育和輔導那些有意願參與的國高中教師,讓他們成為我們合作的種子教師。我們的具體做法是讓這些種子教師將我們的proglearn程式教學系統融入到他們日常的教學實踐中。
  \par 透過這種方式,我們可以收集學生和教師的實際反饋,並進行使用案例和教學成效的數據分析與評估。這一階段的主要目標是建立更多的使用案例,同時不斷對系統進行優化和改進。我們期望能夠提供潛在投資人可靠的參考依據,並使我們的系統更加完善。
  \item 專案執行階段(2026年1月至2026年12月)
  \par 在專案執行階段,我們在已建立合作關係的學校與教師基礎上進行了進一步的洽談。這些種子教師已經具有我們系統的使用經驗,並且發現在教學中難以或無法脫離我們系統的輔助。在這個階段,我們透過這些教師與學校進行合作協商,希望以軟體專案或校園採購案的形式,在校內架設起學校專屬的程式教學系統,並與學校建立起B2B的商業關係。
  \par 同時,我們也將走訪其他學校和資訊補習班,向他們展示我們在先前階段建立的使用案例,探索與他們建立合作關係的可能性。透過這種方式擴大我們的合作規模,尋找新的潛在投資人和合作夥伴。
  \par 這一階段的目的,在於鞏固我們與學校和教師之間的合作關係,並擴大我們的市場範圍和影響力。通過與更多的教育機構建立合作夥伴關係,我們將更有效地推廣我們的產品。未來,我們的目標是參與教育部校園數位內容與教學軟體的公開徵求活動。
\end{enumerate}