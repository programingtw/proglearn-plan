\section{結論與投資效益}

\subsection{營運計畫之結論}

在面對台灣資訊科技領域蓬勃發展,以及數位化學習的潮流下,普羅程式致力於改善程式教育環境,為教育者提供技術支援、開發教學工具,並推出ProgLearn程式教學系統。透過三個階段的營運模式,我們逐步建立起完善的系統服務與客戶群。

我們的營運模式分為自辦階段、種子輔導階段和專案執行階段。在自辦階段,我們舉辦教學活動並與學校建立初步合作關係。種子輔導階段則是培育有意願參與的國高中教師,讓他們成為我們的種子教師,並建立更多的使用案例。最後,在專案執行階段,我們與學校和教師進行合作,建立起交易關係,分別以B2B(軟體專案設計)和B2C(系統訂閱服務)的商業模式進行收費。

透過這三階段的營運模式,我們將逐步完善系統功能並蒐集使用者的使用案例。這些案例不僅在社群媒體上被推廣,更能作為潛在客戶參考,期望能擴大合作的網路並吸引新的投資者和合作夥伴。

\subsection{效益說明}

數位化學習以及智慧學習軟體系統的市場需求持續增長,普羅程式致力於提供軟體上的技術支援與相關的教學工具,協助學校及補習班實現數位轉型,以符合未來的趨勢。其數位化學習的靈活性也有助於應對各種挑戰,如疫情、遠距教學、偏鄉教育以及線上課程需求。

此外,普羅程式針對基層教師的問題,提出了ProgLearn程式教學系統及專案軟體設計方案,目的在改善教育資源的整合與優化、滿足個性化學習需求、解決即時互動的困難以及減輕教師教學負擔等問題。其理念契合聯合國永續發展目標的第四項─優質教育,同時其低硬體需求的特性有助於促進偏鄉教育的發展。

\subsection{潛在風險}

面對教育市場的競爭,普羅程式可能面臨一些挑戰。市場上存在著眾多競爭對手,包括已建立的教育公司以及新興的初創企業如CodingBar等,他們可能擁有更豐富的資源和更廣泛的市場影響力,這將增加普羅程式在市場上建立品牌聲譽和吸引客戶的難度。

另一個潛在風險是缺乏市場信任。教育領域是一個高度敏感且需要建立信任的領域,學校和教育機構可能會對新進入市場的教育科技公司持保留態度,尤其是當普羅程式缺乏充分的市場口碑和客戶評價時,這可能會限制其業務的發展。

為了因應這些風險,普羅程式在過去三年持續利用社群媒體和官方網站推廣科普文章,並開辦了高中課程。目前,我們正積極透過三階段的營運模式,逐步建立與客戶的信任和合作關係。我們的目標是透過實際案例來強化可信度,減少品牌聲譽和市場口碑的影響。同時,我們也致力於從基層開始建立與客戶的合作關係,加強市場的接受度。