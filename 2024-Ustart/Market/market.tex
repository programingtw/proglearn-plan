\section{市場競爭與分析}

\subsection{市場特性與規模}

根據數位發展部數位產業署、資策會與全球科技教育調查研究機構HolonIQ合作發布的2022年台灣智慧學習報告中提及,台灣智慧學習產業的產值達到5,762.9億元新台幣。在這一市場中,軟體系統以其378.8億元新台幣的市場份額,占產值的6.6\%。市場的主要客群涵蓋企業、學校和個人,分別占比33\%、28\%和23\%,這一分布突出了不同領域對智慧學習解決方案的廣泛需求。

軟體系統的增長顯示了教學與學習方式的創新與轉型。從2020年的176.1億元到2022年的378.8億元,教學軟體系統的市場規模呈現出顯著的增長,年增長率達到51.4\%。這一增長趨勢不僅體現了技術進步的速度,也反映了市場對於高效、靈活學習工具的持續需求。

軟體系統分為整合性平台和工具系統兩大類,旨在提升教育品質與效率。整合性平台如學習管理系統(LMS、LCMS)、直播平台、媒合平台及社群平台等,為學習者、教師和教育機構提供全面的教學支持。而工具系統則包括教學內容的數位化工具、平臺架設系統、大數據分析等,這些工具系統強調了技術在教學創新中的應用。

隨著混合式學習平台需求的增加,特別強調線上與實體教學模式的結合,這種模式允許教育活動同時在實體空間和線上進行,從而提高學習的靈活性和可及性。這一趨勢不僅顯示了科技在推動教育創新方面的潛力,也為企業和創業者提供了豐富的市場機會,促使教育方法向更加個性化和高效的方向發展。

\begin{table}[H]
  \begin{tikzpicture}
    \begin{groupplot}[
      group style={
        group size=2 by 1, % 2列1行
        horizontal sep=80pt, % 水平間距
      },
      width=0.43\textwidth, % 每個圖的寬度
    ]

    \nextgroupplot[
      title={教學軟體系統成長趨勢 (2020-2022)},
      xlabel={年份},
      ylabel={市場規模 (億元)},
      xmin=2019, xmax=2023,
      ymin=150, ymax=420,
      symbolic x coords={2019, 2020, 2021, 2022, 2023},
      xtick={2020, 2021, 2022},
      legend pos=north west,
      ymajorgrids=true,
      grid style=dashed,
      nodes near coords,
    ]
    \addplot[
      color=blue,
      mark=square,
    ]
    coordinates {
      (2020,176.1)(2021,316.3)(2022,378.8)
    };

    \nextgroupplot[
      title={主要客群市場份額 (2022)},
      xlabel={客群類型},
      ylabel={市場份額 (\%)},
      symbolic x coords={企業,學校,個人},
      xtick=data,
      ymin=0, ymax=40,
      ybar,
      bar width=20pt,
      legend pos=north east,
      ymajorgrids=true,
      grid style=dashed,
      nodes near coords,
    ]
    \addplot[
      fill=red,
    ]
    coordinates {
      (企業,33)(學校,28)(個人,23)
    };

    \end{groupplot}
  \end{tikzpicture}
  \caption[台灣教學軟體系統市場規模與客群份額]{台灣教學軟體系統市場規模與客群份額 \\ (資料來源:2022 年台灣智慧學習報告)}
\end{table}

% \subsection{市場區隔}

% 我們的工具專為國高中的資訊教育老師設計。
% 這些教師面臨著特定的挑戰,包含大量的備課時間、激發學生對資訊科技的興趣,以及如何評估學生的學習進度。
% 考量到資訊科學教育的特殊性,我們的目標客戶是那些尋求創新、互動式學習工具的老師,他們希望通過這些工具提升學生的學習效果,同時也簡化自己的教學備課和評估過程。

% 資訊教育要求學生不僅要理解理論知識,還要能夠實際應用。
% 然而,現有的教學工具往往無法提供足夠的練習機會,或者缺乏有效的學習動機機制。
% 我們的工具通過提供及時反饋、互動式練習,來解決這些問題,
% 使得教師能夠更有效地教授資訊科學,同時也讓學生在學習過程中更加投入和有成就感。

% 我們的一大特色 -- 互動式講義,他結合教材、筆記、練習與一身。

% \subsection{競爭分析}