\section{產品服務與內容}

\subsection{產品與服務內容}

普羅程式將提供程式教學工具的開發與服務,以及產品ProgLearn程式教學系統。

ProgLearn程式教學系統,以程式教師為中心設計,強調師生間的即時互動,提供低延遲直播、課程與作業管理,以及具有即時反饋功能的數位儀表板、講義視覺化編輯、智慧引導和自動批改等多功能(圖\ref{fig:Classroom})。

\begin{figure}[H]
  \begin{subfigure}{0.5\linewidth}
    \centering
    %   \href{https://raw.githubusercontent.com/programingtw/proglearn-plan/main/img/student.png}{ 
    \includegraphics[width=1\textwidth]{images/student.png}
    %   }
    \caption{學生課堂頁面}
  \end{subfigure}
  \begin{subfigure}{0.5\linewidth}
    \centering
        % \href{https://raw.githubusercontent.com/programingtw/proglearn-plan/main/img/teacher.png}{ 
    \includegraphics[width=1\textwidth]{images/teacher.png}
        % }
    \caption{教師課堂頁面}
  \end{subfigure}
  \caption{課堂介面}
  \label{fig:Classroom}
\end{figure}

\subsubsection{產品的主要介面}

\begin{enumerate}
  \setlength{\parindent}{2em}

  \item 學生課堂頁面:學生的課堂頁面有直播區、互動區、功能區
  \begin{itemize}
    \item 直播區:位於頁面左上用於顯示章節投影片,會在課中展示與老師相同的投影片畫面,並同步老師的滑鼠軌跡、繪畫等。
    \item 互動區:位於頁面右側用於顯示滾動式的講義,講義可以放文字、圖片、課堂習題,並且在課中具有引導功能,會根據老師目前的上課投影片,用黃色框線在講義中顯示其對應的位置。
    \item 功能區:位於頁面左下用於控制直播區的內容,在課中能夠切換投影片、一鍵回到老師的直播投影片等(圖\ref{fig:student-inClass}紅色框線處)。在課後能夠拖動時間軸,回放過去的上課直播(圖\ref{fig:student-review}紅色框線處)。
  \end{itemize}

  \begin{figure}[H]
    \begin{subfigure}{0.5\linewidth}
      \centering
      %   \href{https://raw.githubusercontent.com/programingtw/proglearn-plan/main/img/student.png}{ 
      \includegraphics[width=1\textwidth]{images/student-In class.png}
      %   }
      \caption{學生課堂中}
      \label{fig:student-inClass}
    \end{subfigure}
    \begin{subfigure}{0.5\linewidth}
      \centering
      %   \href{https://raw.githubusercontent.com/programingtw/proglearn-plan/main/img/teacher.png}{ 
      \includegraphics[width=1\textwidth]{images/review.png}
      %   }
      \caption{學生課後回放}
      \label{fig:student-review}
    \end{subfigure}
    \caption{學生課堂頁面}
  \end{figure}

  \item 教師課堂頁面:教師的課堂頁面與學生的課堂頁面相同,同樣有直播區、互動區、功能區,但功能有部分差異。
  \begin{itemize}
    \item 直播區:位於頁面左上用於顯示章節投影片,在課中會將畫面同步到學生的直播區中。
    \item 互動區:位於頁面右側用於顯示滾動式講義,並能夠預覽課堂習題的作答統計。
    \item 功能區:位於頁面左下用於控制直播功能,能夠開啟與關閉直播,在課中可以切換投影片、切換成畫筆、開啟或關閉麥克風功能等。
  \end{itemize}

  \begin{figure}[H]
    \centering
    %   \href{https://raw.githubusercontent.com/programingtw/proglearn-plan/main/img/teacher.png}{ 
    \includegraphics[width=0.7\textwidth]{images/teacher.png}
    %   }
    \caption{教師課堂頁面}
  \end{figure}

  \item 講義編輯頁面:教師可以在該頁面編輯講義的內容,並且將講義的不同部分組合成完整的講義內容(圖\ref{fig:edit-flowchart})。
  \begin{itemize}
    \item 編輯區:在右上點擊不同類型的講義區塊,如文字、選擇題、程式題,就能夠在左上編輯其中的內容。
    \item 腳本區:下半部的時間軸,用於擺放不同類型的講義小區塊。時間軸的單位是投影片的頁數,讓投影片能對應到不同的講義區塊,在課中就能根據投影片的頁數在講義上做引導與提示。
  \end{itemize}
  \begin{figure}[H]
    \centering
    %   \href{https://raw.githubusercontent.com/programingtw/proglearn-plan/main/img/course.png}{ 
    \includegraphics[width=0.7\textwidth]{images/edit.png}
    %   }
    \caption{講義編輯頁面}
  \end{figure}

  \begin{figure}[H]
    \centering
    \begin{tikzpicture}[node distance=4cm]
  
      \node (start) [startstop] {點擊加號新增互動式方塊};
      \node (pro1) [process, right of=start, xshift=2cm] {點擊互動式方塊};
      \node (pro2) [process, right of=pro1] {編輯方塊內容};
      \node (pro3) [process, below of=pro2] {儲存變更};
      \node (pro4) [process, left of=pro3] {拖曳方塊到投影片};
      \node (pro5) [process, left of=pro4] {調整方塊長度};
      \node (stop) [startstop, left of=pro5] {完成講義編輯};
  
      \draw [arrow] (start) -- (pro1);
      \draw [arrow] (pro1) -- (pro2);
      \draw [arrow] (pro2) -- (pro3);
      \draw [arrow] (pro3) -- (pro4);
      \draw [arrow] (pro4) -- (pro5);
      \draw [arrow] (pro5) -- (stop);
  
    \end{tikzpicture}
    \caption{講義編輯的使用流程圖}
    \label{fig:edit-flowchart}
  \end{figure}

\end{enumerate}

\subsubsection{直播功能}

\begin{figure}[H]
  \centering
  %   \href{https://raw.githubusercontent.com/programingtw/proglearn-plan/main/img/live.png}{ 
  \includegraphics[width=0.7\textwidth]{images/streaming.png}
  %   }
  \caption{直播功能區}
\end{figure}

直播功能區位在課堂介面的左半部。此直播功能不同於往常的影像傳輸,而是記錄教師在投影片上的所有操作,包含換頁、繪畫、游標軌跡和聲音實時同步到學生端的介面上,然後存儲在直播記錄中,方便回顧與學習(圖\ref{fig:streaming-flowchart})。
因此相比於Zoom、Meet等視訊會議工具,具有更低的網路延遲與頻寬需求,並且更加穩定。

\begin{figure}[H]
  \centering
  \begin{tikzpicture}[node distance=4cm]

    \node (start) [startstop] {教師移動滑鼠游標};
    \node (pro1) [process, right of=start, xshift=1cm] {將滑鼠移動資訊傳到伺服器};
    \node (pro2) [process, right of=pro1, xshift=2cm] {伺服器儲存至直播記錄};
    \node (pro3) [process, below of=pro2] {伺服器將資訊傳到學生端};
    \node (stop) [process, left of=pro3, xshift=-2cm] {學生端顯示滑鼠游標軌跡};
    \draw [arrow] (start) -- (pro1);
    \draw [arrow] (pro1) -- (pro2);
    \draw [arrow] (pro2) -- (pro3);
    \draw [arrow] (pro3) -- (stop);

  \end{tikzpicture}
  \caption{直播的傳輸流程圖}
  \label{fig:streaming-flowchart}
\end{figure}

由於其低延遲的特性,可以讓教師在即時的課堂環境中使用,作為教學的輔助工具。在線上與偏鄉教學中,其更低的硬體需求,可以作為主要的教學工具。

此外,該直播系統是基於投影片的直播,因此學生可以在課堂中自由地回顧投影片的內容,並且可以在課後自由地拖動時間軸,回放過去的上課直播。

\subsubsection{互動式講義}

\begin{figure}[H]
  \begin{subfigure}{0.5\linewidth}
    \centering
    %   \href{https://raw.githubusercontent.com/programingtw/proglearn-plan/main/img/student.png}{ 
    \includegraphics[width=0.5\textwidth]{images/side-s.png}
    %   }
    \caption{學生課堂頁面}
    \label{fig:student}
  \end{subfigure}
  \begin{subfigure}{0.5\linewidth}
    \centering
        % \href{https://raw.githubusercontent.com/programingtw/proglearn-plan/main/img/teacher.png}{ 
    \includegraphics[width=0.5\textwidth]{images/side-t.png}
        % }
    \caption{教師課堂頁面}
    \label{fig:teacher}
  \end{subfigure}
  \caption{互動式講義區塊}
\end{figure}

互動式講義位在課堂介面的右半部。目的是在課堂中能提供師生間互動的橋樑,並作為教學與學習的輔助工具。其相關功能可細分為以下四點:

\begin{enumerate}
  \setlength{\parindent}{2em}

  \item 智慧引導
  \par 透過淺黃色區塊,向學生指引出當前直播對應到講義上的哪些部分(圖\ref{fig:student})。
  
  \item 即時反饋(數位儀表板)
  \par 為教師提供學生的課堂作答情況,即時了解學生的學習狀況(圖\ref{fig:teacher})。

  \item 自動批改
  \par 為學生提供課堂上的程式作答功能,其使用流程可參考圖\ref{fig:problem-flowchart}。該功能可以自動並即時批改課堂習題(圖\ref{fig:problem}),並且在教師的數位儀表板上顯示學生的作答情況(圖\ref{fig:teacher})。其技術也應用於作業管理中(圖\ref{fig:course}、圖\ref{fig:homework})。
  
  \begin{figure}[H]
    \centering
    \begin{tikzpicture}[node distance=4cm]
  
      \node (start) [startstop] {學生撰寫程式習題};
      \node (pro1) [process, right of=start] {學生點擊submit};
      \node (pro2) [process, right of=pro1, xshift=1cm] {顯示作答結果(圖\ref{fig:problem})};
      \node (pro3) [process, below of=pro2] {教師課堂頁面更新};
      \node (stop) [process, left of=pro3, xshift=-2cm] {數位儀表板顯示作答統計(圖\ref{fig:teacher})};
  
      \draw [arrow] (start) -- (pro1);
      \draw [arrow] (pro1) -- (pro2);
      \draw [arrow] (pro2) -- (pro3);
      \draw [arrow] (pro3) -- (stop);
  
    \end{tikzpicture}
    \caption{程式作答的使用流程圖}
    \label{fig:problem-flowchart}
  \end{figure}

  \begin{figure}[H]
    \begin{subfigure}{0.5\linewidth}
      \centering
      % \href{https://raw.githubusercontent.com/programingtw/proglearn-plan/main/2023全國大專校院智慧創新暨跨域整合創作競賽/img/problem.png}{
        \includegraphics[width=1\textwidth]{images/problem.png}
      % }
      \caption{作答前}
    \end{subfigure}
    \begin{subfigure}{0.5\linewidth}
      \centering
      % \href{https://raw.githubusercontent.com/programingtw/proglearn-plan/main/2023全國大專校院智慧創新暨跨域整合創作競賽/img/problem2.png}{
        \includegraphics[width=1\textwidth]{images/problem-ac.png}
      % }
      \caption{作答後}
    \end{subfigure}
    \caption[互動式講義的程式題]{互動式講義的程式題:點擊下方選項後,會自動將選項插入到程式碼中(紅色箭頭處)。提交後會以不同的顏色框線即時顯示結果,綠色為作答正確,紅色為作答錯誤。}
    \label{fig:problem}
  \end{figure}

  \item 講義視覺化編輯
  \par 為教師提供編輯講義的功能。為實現投影片和講義內容的智慧引導,我們參考了影片剪輯軟體的設計思路,採用了時間軸的概念。將講義分成不同類型的小區塊,並在時間軸上將這些區塊組合成完整的講義內容(圖\ref{fig:edit})。
  \par 在編排講義時,教師可以輕鬆地將這些區塊拖曳到時間軸中。腳本區的時間軸對應著投影片的頁數,因此講義的不同區塊能夠與投影片緊密連接(圖\ref{fig:time})。在課堂中,當老師切換到特定的頁數時,就能自動在講義中引導學生目前的上課內容。

  \begin{figure}[H]
    \begin{subfigure}{0.5\linewidth}
      \centering
      %   \href{https://raw.githubusercontent.com/programingtw/proglearn-plan/main/img/list.png}{ 
      \includegraphics[width=1\textwidth]{images/timezone.png}
      %   }
      \caption{講義編輯邏輯}
      \label{fig:time}
    \end{subfigure}
    \begin{subfigure}{0.5\linewidth}
      \centering
      %   \href{https://raw.githubusercontent.com/programingtw/proglearn-plan/main/img/course.png}{ 
      \includegraphics[width=1\textwidth]{images/edit.png}
      %   }
      \caption{講義編輯頁面}
      \label{fig:edit}
    \end{subfigure}
    \caption{講義編輯功能}
  \end{figure}

\end{enumerate}

\subsubsection{課程與作業管理}

該系統在課堂外,提供了課程與作業管理的功能(圖\ref{fig:course}、圖\ref{fig:homework})。教師可以在此設定課程章節、作業、測驗、考試與查看學生作答狀況。學生可以在此查看課程資訊、撰寫作業並得到即時批改。

\begin{figure}[H]
  \begin{subfigure}{0.5\linewidth}
    \centering
    %   \href{https://raw.githubusercontent.com/programingtw/proglearn-plan/main/img/list.png}{ 
    \includegraphics[width=0.75\textwidth]{images/chapter.png}
    %   }
    \caption{課程章節}
  \end{subfigure}
  \begin{subfigure}{0.5\linewidth}
    \centering
    %   \href{https://raw.githubusercontent.com/programingtw/proglearn-plan/main/img/course.png}{ 
    \includegraphics[width=0.75\textwidth]{images/course.png}
    %   }
    \caption{課程清單}
  \end{subfigure}
  \caption{課程相關頁面}
  \label{fig:course}
\end{figure}

\begin{figure}[H]
  \begin{subfigure}{0.5\linewidth}
    \centering
    %   \href{https://raw.githubusercontent.com/programingtw/proglearn-plan/main/img/list.png}{ 
    \includegraphics[width=0.75\textwidth]{images/homework.png}
    %   }
    \caption{作業作答}
  \end{subfigure}
  \begin{subfigure}{0.5\linewidth}
    \centering
    %   \href{https://raw.githubusercontent.com/programingtw/proglearn-plan/main/img/course.png}{ 
    \includegraphics[width=0.75\textwidth]{images/feedback.png}
    %   }
    \caption{作業反饋}
  \end{subfigure}
  \caption{作業相關頁面}
  \label{fig:homework}
\end{figure}

\newpage
\subsection{營運模式}

普羅程式將透過\ref{sec:plan}節(實施方式、時程規劃及預期成效)所提及的自辦、種子輔導、專案執行三個階段的營運模式,逐步建立起完善的系統服務與客戶群。

% 流程圖:與學校建立合作關係 -> 增加使用案例 -> 與學校建立B2B商業關係
\begin{figure}[h]
  \centering
  \begin{tikzpicture}[node distance=4cm]
  
      % 第一排
      \node (start) [startstop] {自辦階段};
      \node (pro1) [process, right of=start, xshift=2cm] {種子輔導階段};
      \node (stop) [process, right of=pro1, xshift=2cm] {專案執行階段};
  
      % 第二排
      \node (start2) [startstop, below of=start, yshift=2cm] {與學校建立合作關係};
      \node (pro12) [process, right of=start2, xshift=2cm] {增加使用案例};
      \node (stop2) [process, right of=pro12, xshift=2cm] {與學校、教師建立起交易關係};
  
      % 箭頭
      \draw [arrow] (start) -- (pro1);
      \draw [arrow] (pro1) -- (stop);
  
      \draw [arrow] (start2) -- (pro12);
      \draw [arrow] (pro12) -- (stop2);
  
  \end{tikzpicture}
  \caption{營運模式的流程圖}
\end{figure}


\begin{enumerate}
  \setlength{\parindent}{2em}
  \item 自辦階段:由開發人員自辦教學活動,以建立與學校的初步合作關係。
  \item 種子輔導階段:培育有意願參與的國高中教師,讓他們成為我們合作的種子教師,並建立更多proglearn程式教學系統的使用案例。
  \item 專案執行階段:在已建立合作關係的學校與教師的基礎上,與他們進行軟體專案的洽談,並以當前的使用案例為參考,尋找新的潛在投資人和合作夥伴。
\end{enumerate}

自辦階段與種子輔導階段的主要目的,是建立起與目標群眾(國高中教師及其學校)的信任與合作關係,以進入教育市場。接著,在專案執行階段,建立起與學校、教師間的交易關係,分別以B2B\footnote{B2B:Business to Business,指企業對企業的商業模式。}與B2C\footnote{B2C:Business to Consumer,指企業對消費者的商業模式。}的商業模式進行收費(詳見\ref{sec:revenue}營收模式)。

\subsection{營收模式} % 收益流
\label{sec:revenue}
針對不同種類的用戶實行不同的收費模式,分為B2B(學校端)與B2C(教師端)兩種收費模式:

\begin{enumerate}
  \setlength{\parindent}{2em}
  \item B2B(學校端)
  \par 針對學校、補教業者等機構提供專案設計與程式教學系統建置的服務。協助機構架設教學系統於自家伺服器上,並提供技術支援。收費模式為一次性專案設計費用與年度授權費用,其專案設計費用依據機構規模與使用人數而定,年度授權費用應收取所有教師及其課程10\%的收益。
  \item B2C(教師端)
  \par 針對個體教師提供教學系統的服務。在我們建置的線上教學系統上,提供教師月訂閱制的付費帳號,並提供技術支援。收費模式為月訂閱制費用,其費用是抽取該教師所有課程5\%的收益。
\end{enumerate}