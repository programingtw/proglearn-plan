\section{行銷策略}

\subsection{目標消費族群}
\setlength{\parindent}{2em}
\par 在近年,政府與教育部積極推動資訊教育,但在我們與教師訪談後發現,資訊教育在實行上仍然有許多問題,主要問題包括:硬體設備不齊全、專業師資不足等,導致學校不重視或無法積極推動資訊教育。我們也有經由網路調查發現,資訊補習班在師生互動性上稍嫌不足,我們的產品便可以為這些問題提供解決方案。
\par 程式教學是我們團隊的強項,我們認為這是解決教育機構資訊教育不足的商機。我們打算與教育機構以及教育工作者合作,提供針對高中職學校、資訊補習班以及教師所缺少的資訊教育能力的教學軟體即服務(SaaS),提供教學者更好的教學工具來輔助上課,使教學者能夠更加輕鬆的準備課程,學生也能有更好的上課體驗,讓普羅大眾都可以輕鬆學習程式,這正是我們發現的商機所在。

\begin{table}[H]
  \centering
  \caption[商業服務模式表]{商業服務模式表}
  \renewcommand{\arraystretch}{1.1} % 增加行高
  \begin{tabular}{|c|c|c|}
    \hline
    \multicolumn{2}{|c|}{服務模式} & \\
    \hline
    \multirow{5}{*}[-0.5ex]{目標消費族群} & 身分 & 
    \begin{minipage}[c]{0.5\columnwidth}
      \begin{enumerate}[label=(\arabic*)]
        \setlength{\parindent}{1em}
        \item 國高中學校
        \item 資訊補習班
        \item 個體教師(家教、線上課程講師)
      \end{enumerate}
    \end{minipage} \\
    \cline{2-3}
    ~ & 性別 & 不限 \\
    \cline{2-3}
    ~ & 收入 & 不限 \\
    \cline{2-3}
    ~ & 居住地區 & 目前限中文用語地區 \\
    \cline{2-3}
    ~ & 擁有共通興趣 &
    \begin{minipage}[c]{0.5\columnwidth}
      \begin{enumerate}[label=(\arabic*)]
        \setlength{\parindent}{1em}
        \item 資訊設備或師資不足的教學機構
        \item 學生對於資訊相關教育有興趣的教育場所
      \end{enumerate}
    \end{minipage} \\
    \hline
    \multirow{5}{*}[-0.5ex]{目標使用族群} & 身分 & 教育者 \\
    \cline{2-3}
    ~ & 性別 & 不限 \\
    \cline{2-3}
    ~ & 收入 & 不限 \\
    \cline{2-3}
    ~ & 居住地區 & 目前限中文用語地區 \\
    \cline{2-3}
    ~ & 擁有共通興趣 & 
      \begin{minipage}[c]{0.5\columnwidth}
        \begin{enumerate}[label=(\arabic*)]
          \setlength{\parindent}{2em}
          \item 想給予學生更多吸收資訊知識的導師
          \item 無法解決學校設備缺乏的困擾
        \end{enumerate}
      \end{minipage} \\
    \hline
  \end{tabular}
\end{table}
\newpage
\subsection{行銷策略}

\begin{enumerate}[label=(\arabic*)]
  \setlength{\parindent}{2em}
  \item 過去的行銷成果
    \par 截至2024年1月15日,普羅程式已於Instagram
    、Facebook
    、普羅官網
    三個社群平台上共有六百多名追蹤者(附件\ref{fig:Appendix-Marketing})。我們於平台上推廣科普文章、普羅課程資訊及資訊時事供社群大眾閱讀,過去更在單月內創造5811名貼文觸及人數(圖\ref{fig:article}),並在高中開辦暑期短期課程(圖\ref{fig:course-summer}),增加品牌信心與知名度。
    
    % \begin{figure}[H]
    %   \centering
    %   \begin{subfigure}{0.32\linewidth}
    %     \centering
    %     \includegraphics[width=0.8\textwidth]{images/FB.png}
    %     \caption{Facebook}
    %     \label{fig:FB}
    %   \end{subfigure}
    %   \begin{subfigure}{0.32\linewidth}
    %     \centering
    %     \includegraphics[width=0.6\textwidth]{images/IG.jpg}
    %     \caption{Instagram}
    %     \label{fig:IG}
    %   \end{subfigure}
    %   \begin{subfigure}{0.32\linewidth}
    %     \centering
    %     \includegraphics[width=0.8\textwidth]{images/website.png}
    %     \caption{普羅官網}
    %     \label{fig:website}
    %   \end{subfigure}
    %   \caption[社群平台]{社群平台}
    %   \label{fig:platform}
    % \end{figure}
    
    \begin{figure}[H]
      \centering
      \begin{subfigure}{0.49\linewidth}
        \centering
        \includegraphics[width=1\textwidth]{images/article.png}
        \caption[科普文章]{科普文章與FB的推廣數據}
        \label{fig:article}
      \end{subfigure}
      \begin{subfigure}{0.49\linewidth}
        \centering
        \includegraphics[width=1\textwidth]{images/course-summer.png}
        \caption{暑期短期課程}
        \label{fig:course-summer}
      \end{subfigure}
      \caption[過去的行銷成果]{過去的行銷成果}
      \label{fig:marketing}
    \end{figure}

  \item 目前的行銷規劃
  	\par 以熟識的老師作為第一批試用目標,目前與我們有合作的高中有東山高中、丹鳳高中、鳳山高中。我們將積極收集老師與學生的回饋,包括他們對工具的使用感受、意見和建議。我們會與這些老師建立密切的合作關係,並請他們分享他們的使用經驗與推廣,以便建立我們的口碑。
  	\par 在增加功能及知名度後,我們計劃在線上透過社群媒體分享示範課程的短影片、教師使用心得等等,在線下透過實地訪談,並提供過去的使用者紀錄及成效,並且透過熟識老師們的口碑推薦,加強我們在目標客戶群體中的知名度,並吸引更多的潛在客戶。
  \item 未來的行銷規劃
  	\par 在完善功能與建立一定知名度後,我們計畫與個人教師合作,如:試用並分享心得、推薦給其他使用者並獲得優惠、擔任合作教師並錄製課程供宣傳、教育機構建立合作夥伴關係等等,提供定制化的解決方案和支持服務。透過與個人教師和教育機構的合作,我們可以更好地了解客戶需求,並持續改進產品功能,同時擴大市場覆蓋範圍。我們也將持續投資於市場營銷和品牌宣傳,提升產品知名度和品牌價值,並提供給客戶優質的產品,同時聆聽客戶的反饋,保持與客戶的互動,提高我們的品牌價值,以吸引更多的客戶和合作夥伴加入我們的生態系統。
\end{enumerate}

\begin{figure}[H]
  \centering
  \begin{tikzpicture}[node distance=1.9cm, auto]
    % Place nodes
    \node [startstop] (start) {過去的成果};
    \node [process, below of=start] (pro1) {建立社群媒體、推廣科普文章、舉辦營隊};
    \node [startstop, below right of=pro1, xshift=1.5cm, yshift=-0.5cm] (pro2) {目前的規劃};
    \node [process, below of=pro2] (pro3) {教師合作試用、分享與推廣};
    \node [startstop, below of=pro3] (pro4) {分享課程短影音、線下訪談};
    \node [startstop, right of=pro4, xshift=3cm] (pro5) {未來的規劃};
    \node [startstop, below right of=pro5, xshift=2cm, yshift=-0.5cm] (pro6) {提升品牌價值};
    \node [startstop, below left of=pro5, xshift=-2cm, yshift=-0.5cm] (pro7) {增加合作夥伴};
    \node [process, below of=pro6] (pro8) {提供優質產品、聆聽回饋、保持互動};
    \node [process, below of=pro7] (pro9) {提供優惠、成為合作教師};

    % Draw edges
    \draw [arrow] (start) -- (pro1);
    \draw [arrow] (pro1) |- (pro2);
    \draw [arrow] (pro2) -- (pro3);
    \draw [arrow] (pro3) -- (pro4);
    \draw [arrow] (pro4) -- (pro5);
    \draw [arrow] (pro5) |- (pro6);
    \draw [arrow] (pro5) |- (pro7);
    \draw [arrow] (pro6) -- (pro8);
    \draw [arrow] (pro7) -- (pro9);
  \end{tikzpicture}
  \caption{行銷策略流程圖}
\end{figure}
