\begin{thebibliography}{99}
  \bibitem{ref:111產業產值調查報告} 數位發展部 數位產業署、財團法人資訊工業策進會。"111年台灣智慧學習產業產值調查報告 - 智慧學習產業整合輸出計畫",民111年。
  \bibitem{ref:110產業產值調查報告} 張筱祺、鐘映庭。"經濟部工業局110年度專案計畫 - 智慧學習產業整合輸出計畫全球輸出網絡佈建分項計畫 - 智慧學習產業產值調查報告。" 財團法人資訊工業策進會產業情報所委託研究報告。資訊工業策進會數位教育研究所,民110年。
  \bibitem{ref:學生數量} 政府資料公開平台(民113年1月16日)。全臺灣各級學校之學生數及畢業生數資料。民113年2月14日,取自:https://data.gov.tw/dataset/31436。
  \bibitem{ref:企業培訓} 陳旻萃。"智慧培訓模式的發展趨勢與應用。" 人事月刊,民105年6月6日,第370期。
  \bibitem{ref:老師的困難} 林文瑛、陳衍宏、周蔚倫。"新冠疫情下線上同步教學演練的啟示:課堂參與程度與課堂環境及學習經驗之關係。" 長庚人文社會學報,14:2(2021),179-214。
  \bibitem{ref:補教業者} 楊文君(2019年9月6日)。程式設計納入正式課綱 家教時薪達2000元以上。取自:https://www.rti.org.tw/news/view/id/2033475
  \bibitem{ref:家教} 秦宛萱(2019年9月7日)。抓住父母望子成龍的心!5成4學生想當家教 程式設計時薪高達2500元。取自:https://www.cmmedia.com.tw/home/articles/17410
\end{thebibliography} 