\section{財務計畫}

\subsection{成本分析}

團隊的主要產品為軟體系統,成本主要集中於人力與設備。在人力成本方面,開發人員薪資為主要支出,此外還需支付行政人員的薪資。設備成本則涵蓋伺服器租用費用以及必要的硬體設備購置。在試辦階段,除了系統開發,我們還將進行內部測試和在合作學校舉辦的自辦課程,這需要支付場地租用費及相關的設施使用費。

進入種子培訓階段,隨著使用者的增加,我們將提高伺服器租用的規模,並投入專利申請和公司設立的費用。市場推廣則著重於品牌宣傳、開設工作坊和產品介面設計。

到了專案執行階段,成本將進一步增加,包括參加教育科技展來提升品牌知名度的市場推廣費用,以及因應商業擴展導致的薪資和伺服器租用費用增長。此外,還會有新的印刷品製作和團隊旅費的開支。

\begin{figure}[htbp]
  \centering
  \begin{minipage}[t]{0.5\textwidth}
    \centering
    \begin{table}[H]
      \caption{普羅程式-開發成本}
      \centering
      \begin{tabular}{|c|c|c|c|}
        \hline
        \thead{會計項目} & \thead{113年度} & \thead{114年度} & \thead{115年度} \\
        \hline
        薪資支出 & (240,000) & (420,000) & (840,000) \\
        \hline
        伺服器租金費 & (30,000) & (150,000) & (300,000) \\
        \hline
        網域費用 & (1,000) & (1,000) & (1,000) \\
        \hline
        設備費 & (50,000) & (60,000) & (120,000) \\
        \hline
        視覺設計費 & (20,000) & (80,000) & (160,000) \\
        \hline
        場地租用費 & (10,000) & (30,000) & (100,000) \\
        \hhline{|=|=|=|=|}
        小計 & (351,000) & (741,000) & (1,521,000) \\
        \hline
      \end{tabular}
    \end{table}
  \end{minipage}
  \hfill
  \begin{minipage}[t]{0.44\textwidth} 
    \centering
    \begin{table}[H]
      \caption{普羅程序-營運成本}
      \centering
      \begin{tabular}{|c|c|c|c|}
        \hline
        \thead{會計項目} & \thead{113年度} & \thead{114年度} & \thead{115年度} \\
        \hline
        薪資支出 & (96,000) & (300,000) & (600,000) \\
        \hline
        成立公司費 & (0) & (15,000) & (0) \\
        \hline
        專利申請費 & (0) & (50,000) & (0) \\
        \hline
        市場推廣費 & (3,000) & (25,000) & (100,000) \\
        \hline
        印刷費 & (5,000) & (20,000) & (100,000) \\
        \hline
        諮詢費 & (60,000) & (60,000) & (80,000) \\
        \hline
        旅費 & (15,000) & (50,000) & (120,000) \\
        \hline
        雜支 & (15,000) & (100,000) & (250,000) \\
        \hhline{|=|=|=|=|}
        小計 & (194,000) & (620,000) & (1,250,000) \\
        \hline
      \end{tabular}
    \end{table}
  \end{minipage}
\end{figure}

\subsection{預期收入}

在我們的業務發展策略中,收入模式根據不同的營運階段有所區別。 以下是未來三年的收入預測,詳細解釋了每個階段的收入來源和預期收益。

\subsubsection{第一階段-試辦課程與收集回饋 (2024Q2-2024Q4)}
在新創階段,我們的主要目標是透過產品測試和市場回饋來調整和優化我們的系統。
由於這是系統引入市場的初期,我們將不會從使用者那裡直接收取費用。
相反,收入將主要來自創業競賽的獎金和孵化器的種子資金。

\subsubsection{第二階段-種子教師訓練 (2025Q1-2025Q3)}
進入第二年,我們將邀請教師免費試用我們的系統,以便收集寶貴的使用者回饋並進一步改進產品。
儘管提供了免費的基礎服務,但我們將透過銷售插件來創造收入,這些插件為教師提供額外的功能和服務。
預計透過成功銷售約200個插件,每個插件定價為100至1000元,從而實現總收入為250,000元。

\subsubsection{第三階段-專案執行 (2025Q4-2026Q4)}
在這一階段,我們將透過兩種主要的收入模式來實現業務的商業化:B2B和B2C。
我們預計將與10所學校合作,為他們提供完整的教學系統解決方案,每所學校支付的年度授權費為100,000元,總計收入為1,000,000元。
此外,我們計劃吸引300名教師透過訂閱制形式使用我們的系統,每位教師的年費為3,000元,總計收入為900,000元。
透過提供額外的插件和服務,我們預計將額外收入150,000元。
教師訂閱將包括一定數量的學生帳戶,若教師需增加更多學生帳戶,我們將收取額外費用,預計為100,000元。

\subsubsection{總結}
從第二年開始,我們將透過插件銷售實現初步的直接收入。
到第三年,隨著與更多學校和教師的合作,我們預計總收入將顯著增加,以實現更穩定和多元化的收益流。
這將有助於我們不僅回收初期投資,還能持續擴展和優化我們的教學解決方案。

\subsubsection{預期損益表}

\begin{table}[H]
  \caption{普羅程式-預期損益表}
  \centering
  \begin{tabular}{|c|c|c|c|}
    \hline
    \thead{會計項目} & \thead{2024年度} & \thead{2025年度} & \thead{2026年度} \\ 
    \hline
    營業收入 & 0 & 637,500 & 2,150,000 \\ 
    \hline
    營業成本 & (545,000) & (1,361,000) & (2,771,000) \\
    \hline
    營業毛利 & (545,000) & (723,500) & (621,000) \\
    \hline
    業外收入\tablefootnote{包括但不限於創業競賽、貸款或輕量級投資。} & 400,000 & 1,000,000 & 2,000,000 \\
    \hhline{|=|=|=|=|}
    稅前盈餘 & (145,000) & 276,500 & 1,379,000 \\
    \hline
  \end{tabular}
\end{table}
