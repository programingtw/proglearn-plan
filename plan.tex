\documentclass[12pt]{article}

\usepackage[right=20mm, left=20mm]{geometry}
\usepackage{type1cm}
\usepackage{amssymb}
\usepackage[fleqn]{amsmath}
\usepackage{tikz}
\usepackage{multicol}
\usepackage{makecell}
\setlength{\columnsep}{1pt}
\usepackage{pgfplots}
\usepackage{float}

\usepackage{indentfirst}
\usepackage{lastpage}  
\usepackage{fancyhdr}
\pagestyle{fancy}

\usepackage[unicode=true,pdfusetitle,
 bookmarks=true,bookmarksnumbered=false,bookmarksopen=false,
 breaklinks=false,pdfborder={0 0 1},backref=false,colorlinks=false]
 {hyperref}

\makeatletter
\newenvironment{myalign*}{\ifvmode\else\hfil\null\linebreak\fi
  \hspace*{-\leftmargin}\minipage\textwidth
  \setlength{\abovedisplayskip}{0pt}%
  \setlength{\abovedisplayshortskip}{\abovedisplayskip}%
  \start@align\@ne\st@rredtrue\m@ne}%
{\endalign\endminipage\linebreak}

% Paper size
\topmargin -10mm
\textwidth 170mm
% \oddsidemargin -5mm
% \evensidemargin -5mm
\textheight 220mm

% Font setting
\usepackage{xeCJK}
% \setCJKmainfont{Noto Sans TC}
\setCJKmainfont{kaiu.ttf}


\renewcommand{\footnotesize}{\normalsize} 
\renewcommand{\headrulewidth}{0pt}
\renewcommand{\footrulewidth}{0pt}

\lhead{}
\chead{整合式線上教學平台ProgLearn}
\rhead{}

\lfoot{}
\cfoot{}
\rfoot{ 共 \pageref{LastPage} 頁 第  \thepage   頁} 

\makeatletter
\begin{document}
% \fontsize{14pt}{18pt}\selectfont
% \author{}
\date{}
\usetikzlibrary{automata, positioning, arrows}
% \maketitle
\tikzset{every state, accepting/.style={double distance=2pt}}

\begin{enumerate}
  \setlength{\parindent}{2em}
  \item 摘要 
    \par 本計畫將建立一個專注於教師導向的教學工具,
    名為ProgLearn,這個整合式教學平台將提供課堂和線上解題系統(Online Judge)。
    課堂將包含影片、互動式講義以及課堂練習。
    影片將提供同步和非同步教學,而互動式講義不再是傳統的圖像及文字,
    將使用JS(JavaScript)技術製作,以增加學生和教師的互動。
    Online Judge將包含傳統的題目撰寫外,還提供自動反饋的功能,
    可以分析學生提交的程式碼,以提高學生的學習效果。
    課堂練習平台將採用Online Judge模組,如果採用同步教學,
    學生提交的程式碼可以即時在課堂練習上呈現,讓教師和學生可以即時互相評論與回饋。
    除此之外,ProgLearn系統還將提供可自訂的測試案例,以方便教師進行練習的設計。
    綜上所述,ProgLearn是一個結合教學與練習的完整教學平台,旨在提高學生的學習效率,
    並改善教師的教學效率。
  \item 研究動機與研究問題
    \par 在台灣,資訊科技領域受到廣泛重視。
    根據十二年國民基本教育課程綱要\cite{ref1},
    程式設計已被列為學校必修課程,每年有數百萬學生修習程式教育。
    然而,在基層教育中卻存在許多問題:
  \item 文獻回顧與探討
  \item 研究方法及步驟
  \item 預期結果
  \item 參考文獻
    \renewcommand{\section}[2]{}
    \begin{thebibliography}{99}  
      \bibitem{ref1} Text 代補
    \end{thebibliography} 
  \item 需要指導教授內容
\end{enumerate}


\end{document}