\documentclass[12pt]{article}

\usepackage{enumitem}
\usepackage[right=20mm, left=20mm]{geometry}
\usepackage{type1cm}
\usepackage{amssymb}
\usepackage[fleqn]{amsmath}
\usepackage{tikz}
\usepackage{multicol}
\usepackage{makecell}
\setlength{\columnsep}{1pt}
\usepackage{pgfplots}
\usepackage{float}
\usepackage{caption}
\usepackage{subcaption}
% \usepackage{subfig}
\usepackage{graphicx}

\usepackage{indentfirst}
\usepackage{lastpage}  
\usepackage{fancyhdr}
\pagestyle{fancy}

\usepackage[unicode=true,pdfusetitle,
 bookmarks=true,bookmarksnumbered=false,bookmarksopen=false,
 breaklinks=false,pdfborder={0 0 1},backref=false,colorlinks=false]
 {hyperref}

\makeatletter
\newenvironment{myalign*}{\ifvmode\else\hfil\null\linebreak\fi
  \hspace*{-\leftmargin}\minipage\textwidth
  \setlength{\abovedisplayskip}{0pt}%
  \setlength{\abovedisplayshortskip}{\abovedisplayskip}%
  \start@align\@ne\st@rredtrue\m@ne}%
{\endalign\endminipage\linebreak}

% Paper size
\topmargin -10mm
\textwidth 170mm
% \oddsidemargin -5mm
% \evensidemargin -5mm
\textheight 220mm

% Font setting
\usepackage{xeCJK}
% \setCJKmainfont{Noto Sans TC}
\setCJKmainfont{kaiu.ttf}


\renewcommand{\footnotesize}{\normalsize} 
\renewcommand{\headrulewidth}{0pt}
\renewcommand{\footrulewidth}{0pt}

\lhead{}
\chead{整合式線上教學平台ProgLearn}
\rhead{}

\lfoot{}
\cfoot{}
\rfoot{ 共 \pageref{LastPage} 頁 第  \thepage   頁} 

\makeatletter
\begin{document}
% \fontsize{14pt}{18pt}\selectfont
% \author{}
\date{}
\usetikzlibrary{automata, positioning, arrows}
% \maketitle
\tikzset{every state, accepting/.style={double distance=2pt}}
\captionsetup[figure]{labelfont={bf},name={圖},labelsep=period}

\noindent
\textbf{參賽隊名:} 普羅程式 \\
\textbf{作品名稱:} 整合式線上教學平台 ProgLearn

\begin{enumerate}
  \setlength{\parindent}{2em}
  \item 作品簡介
  \begin{enumerate}
    \setlength{\parindent}{2em}
    \item 創作動機與背景
    \par 在台灣,資訊科技領域受廣泛重視,且程式設計已被列為學校必修課程\cite{ref2}。
    儘管每年有數百萬學生修習程式課程\cite{ref3},但是在基層教育中卻存在許多問題。
    \item 作品目的
    \par 本計畫將建立一個名為 ProgLearn 的教學工具,專注於教師導向的教學工具,讓所有人都能做程式教學。
    \item 研究問題與專業領域
    \par 根據教師的教學需求,整理出四個主要問題:
    \begin{enumerate}[label=(\arabic*)]
      \setlength{\parindent}{2em}
      \item 教師的教學有諸多限制:
        \par 程式設計是實作很重要的課程,常見的線下教學方式需要使用廣播與管理系統,但這些系統都需要在教師授課時控制學生的畫面,就導致老師需要頻繁切換畫面給學生練習,這種情況下教師就無法流暢教學,影響教學品質。
      \item 教師需要花多時間準備上課的環境:
        % \par 教師需要在上課前,準備線上課程的直播環境、準備課程教材、批改學生作業、上傳考題、課外時間回覆學生的各種問題等\cite{ref4}。
        \par 在目前的程式教學環境中,教師需要花費相當多的時間來準備上課所需的環境。包括設定各種軟體與工具,不但要使它們正確運作,還要確保學生能夠順利使用這些工具。
        然而,這樣的準備過程往往是費時的。教師需要在每堂課前花費大量時間來設置不同的軟體和平台,例如 Google Meet 和程式作業的繳交系統。這些工作需要教師在每次上課前進行,耗費了他們寶貴的教學時間。
        此外,教師還需要處理與環境相關的技術問題,例如軟體更新、安裝問題、網路連接等等。這些問題都可能延誤教學過程,進一步增加了教師準備上課環境的時間。
      \item 教師無法即時得知學生的狀況:
        \par 傳統的程式教學中,教師需要從課後的程式作業中,才能了解學生遇到的問題,而大多數情況下,學生也沒有適當的管道跟契機能跟教師反饋。由於教師無法即時在課堂上得知學生的狀況,就難以保證教學的內容品質,進而影響教學成效\cite{ref7}。
    \end{enumerate}
    \par 基於以上問題,本系統跨足教育領域,通過改進教師的教學方式,幫助教師以最小的成本實現預期的教學效果,從而提高學生的學習體驗和效果。
    \item 創新創意
      \par 本系統為符合教師的教學需求,設計出以下幾種創新功能:
      \begin{enumerate}[label=(\arabic*)]
        \setlength{\parindent}{2em}
        \item 課堂直播
          \begin{itemize}
            \item 即時同步的直播投影片:此直播系統不同於往常的影像傳輸,而是記錄教師在投影片上的所有操作。包含換頁、繪畫、游標軌跡和聲音實時同步到學生端的介面上,然後存儲在直播記錄中,方便回顧與學習。
            \item 可隨時切換的投影片:為了滿足每位學生的學習需求,我們讓學生在直播過程中,按照自己的學習節奏,切換到其他簡報,隨時都可以迅速返回到老師的直播投影片。
            \item 直播回放:學生可以在課後回放老師的直播記錄,並且能夠隨時切換到其他簡報,方便學生複習。
            \item 我們與其他平台的差異性:與Zoom、Google Meet等以視訊為主的平台相比,我們的平台的特色是讓學生能夠操控投影片,以此回顧上課內容,這是他們所缺乏的。我們的平台更注重互動性和使用者的學習體驗,讓學生有更大的學習彈性與自由度。
          \end{itemize}
        \item 互動式講義
          \begin{itemize}
            \item 滾動式講義設計:教師可使用 Markdown 來編輯和呈現課程講義。其滾動式的設計不僅去除了學生在查看講義時可能遇到的頁面限制,使學生在瀏覽上更加順暢。配合教學跟隨模式,我們的系統會透過淺黃色區塊提示學生,老師當下的教學內容,以提供更高效的學習體驗。
            \item 嵌入式的程式練習題:可以將程式練習題嵌入至講義的內容中,讓學生可以直接作答,並即時得知答案是否正確以及錯誤的原因。此外,教師也能從教學頁中得知學生的作答狀況,以便於調整教學內容。
            \item 生成式講義:利用 AI 技術分析教師上傳的投影片,根據投影片的內容自動生成補充講義的草稿,並且還能生成程式練習題,讓教師能夠更快速的準備課程,提供多元化的教學方式。
          \end{itemize}
          
          \begin{figure}[H]
            \begin{subfigure}{0.45\linewidth}
              \centering
              \href{https://raw.githubusercontent.com/programingtw/proglearn-plan/main/img/interactiveMeterial.png}{ 
                \includegraphics[width=0.8\textwidth]{./img/interactiveMeterial.png}
              }
              \caption{互動式講義學生介面}
              \label{active}
            \end{subfigure}
            \begin{subfigure}{0.45\linewidth}
              \centering
              \href{https://raw.githubusercontent.com/programingtw/proglearn-plan/main/img/interactiveMeterial2.png}{ 
                \includegraphics[width=0.65\textwidth]{./img/interactiveMeterial2.png}
              }
              \caption{程式練習題的教師回饋介面}
              \label{active2}
            \end{subfigure}
            \caption{互動式講義 (點擊可看大圖)}
          \end{figure}
        %\par 使用情境:演算法課程中,老師剛上完搜尋演算法並出了一個習題,此習題是搜尋程式碼的片段並挖了幾個空格,學生填答空格後,老師能即時取得學生的填答狀況。(同步回答圖)
      \end{enumerate}
      \newpage
    \item 主要功能
    \par 該系統提供課堂教學系統和課後作業系統。
    課堂教學系統提供投影片直播、投影片繪畫、互動式講義、學生的答題統計的功能。
    課後作業系統提供自動批改程式碼作業的功能。
    \item 實用性與預期貢獻
    \par 協助教師提高教學效率,並且提高學生的學習效果。在未來能將此系統提供給如學校、補習班、講座等老師使用。
  \end{enumerate}

  \item 需求分析
    \par 本計畫將依循軟體工程流程與方法進行需求擷取與分析、架構與介面設計。
    \begin{enumerate}
      \setlength{\parindent}{2em}
      \item 本系統主要的功能性需求:%以itemindent調整縮排,但應該有更好的做法能一同縮排。
        \begin{enumerate}[itemindent=12pt, label=\Alph*.]
          % \item 班級與課程功能
          %   \begin{enumerate}[itemindent=24pt, label=\Alph{enumiii}-\arabic*.]
          %     \item 教師可新增與查詢自己的課程,並於特定課程中新增章節。
          %     \item 教師可於特定章節中查看此章節中,學生的答題統計、問答紀錄。
          %   \end{enumerate}
          \item 直播投影片功能
            \begin{enumerate}[itemindent=24pt, label=\Alph{enumiii}-\arabic*.]
              \item 教師可於特定章節中上傳課程投影片,並在投影片中實時繪畫、操作。
              \item 教師的所有投影片操作(換頁、繪畫、游標軌跡)和聲音都將實時同步到學生端的介面上,並存儲在直播記錄中。
              \item 學生可於課後回放老師的直播記錄。
            \end{enumerate}
          \item 教學頁面功能
            \begin{enumerate}[itemindent=24pt, label=\Alph{enumiii}-\arabic*.]
              \item 學生可於投影片區觀看並操作投影片。
              \item 學生可於程式區編輯 JS 程式碼,並且能執行與查看結果。
              \item 教師可查看各學生執行的程式碼與結果。
              \item 教師可於直播區進行教學,並將操作同步到所有學生端的介面上。
              \item 學生可於講義區讀取教師編輯的課程講義,並可進行程式練習。
              \item 教師可於答題區觀看學生的答題情況,並可針對答題結果給予回饋。
            \end{enumerate}
          \item 互動式講義功能
            \begin{enumerate}[itemindent=24pt, label=\Alph{enumiii}-\arabic*.]
              \item 教師可使用 Markdown 來編輯和呈現課程講義,並可在講義中嵌入程式練習題。
              \item 教師可開啟教學跟隨模式,系統會透過淺黃色區塊提示學生,老師當下的教學內容。
              \item 學生可在講義中直接回答嵌入的程式練習題,並即時得知答案是否正確以及錯誤的原因。
            \end{enumerate}
          \item 答題與反饋功能
            \begin{enumerate}[itemindent=24pt, label=\Alph{enumiii}-\arabic*.]
              \item 教師可於特定章節中新增題目,並且可以在現有題目系統(如: AtCoder%$\footnote{https://atcoder.jp/}$
              )直接抓取題目。
              \item 教師可於特定章節中查看各學生於特定作業的答題情況,內容包含繳交的程式碼與分數。
              \item 教師可針對答題情況,留下教師建議。
              \item 學生可於特定章節中查看並解決此章節的課後練習題目。
            \end{enumerate}
          \end{enumerate}
      \item 非功能性需求
        \begin{enumerate}[itemindent=12pt]
          \item [A.] 使用者介面與人為因素
          \begin{enumerate}[itemindent=24pt]
            \item [A-1.] 使用者:程式設計課程的老師、學生
            \item [A-2.] 使用者的介面設計:簡單並且易上手、高度的功能整合、即時獲得使用者的反饋與資訊。
            \item [A-3.] 使用者的引導與教學:直觀的UI設計並且對所有老師做系統的使用培訓。
          \end{enumerate}
          \item [B.] 設備
            \begin{enumerate}[itemindent=24pt]
              \item [B-1.] 使用者的使用設備:使用設備以電腦為主、手持設備為輔,並針對電腦使用最佳化。
              \item [B-2.] 使用者的設備限制:所有能瀏覽網頁的設備皆可使用。
            \end{enumerate}
          \item [C.] 效能
            \begin{enumerate}[itemindent=24pt]
              \item [C-1.] 反應時間:同一課程能讓100人同時使用,並在0.5秒內回應使用者,以保障使用者的上課體驗。
              \item [C-2.] 容量限制:限制每位教師在單一章節中,只能上傳20MB以下的投影片。
              \item [C-3.] 課堂限制:每位教師的課程需要經過管理員審核後才能開課,並且每位教師以開設5門課為限。
              \item [C-4.] 課堂人數:同一課程最多只能有100位學生。
            \end{enumerate}      
          \item [D.]錯誤處理:
            \begin{enumerate}[itemindent=24pt]
              \item [D-1.] 系統遇到不正常的負載:針對大量請求的用戶限制封包的流量。
              \item [D-2.] 系統遇到高負載:使用排隊來限制同時登入人數。
            \end{enumerate}
        \end{enumerate}
      \item 系統設計
      \par 本系統是架設於網路上的 Web 應用系統,以下將針對系統的設計分為系統架構、系統介面說明:
        \begin{enumerate}[label=(\arabic*)]
        \setlength{\parindent}{2em}
          \item 系統架構
          \par 根據需求擷取與分析所述,系統架構如圖\ref{arc1}所示,
          在 Proglearn 系統內(如圖\ref{arc1}),
          包括前端服務器(Web Application Server、Single Page Application)、
          主要後端服務器(Business Backend)、
          課程管理服務器(Course Manage Server)、
          線上解題服務器(Online Judge Server)、
          資料庫(Database)等容器。
          \begin{figure}[htb]
            \centering
            \begin{subfigure}{0.45\linewidth}
              \centering
              \href{https://raw.githubusercontent.com/programingtw/proglearn-plan/main/img/arc1.jpg}{
                \includegraphics[width=0.65\textwidth]{./img/arc1.jpg}
              }
              \caption{Proglearn 系統架構}
              \label{arc1}            
            \end{subfigure}
            \begin{subfigure}{0.45\linewidth}
              \centering
              \href{https://raw.githubusercontent.com/programingtw/proglearn-plan/main/img/arc2.jpg}{
                \includegraphics[width=0.65\textwidth]{./img/arc2.jpg}
              }
              \caption{主要後端服務器架構}
              \label{arc2}
            \end{subfigure}
            \bigskip
            \begin{subfigure}{0.45\linewidth}
              \centering
              \href{https://raw.githubusercontent.com/programingtw/proglearn-plan/main/img/arc3.jpg}{
                \includegraphics[width=0.65\textwidth]{./img/arc3.jpg}
              }
              \caption{課程管理服務器架構}
              \label{arc3}
            \end{subfigure}
            \begin{subfigure}{0.45\linewidth}
              \centering
              \href{https://raw.githubusercontent.com/programingtw/proglearn-plan/main/img/arc4.jpg}{
                \includegraphics[width=0.65\textwidth]{./img/arc4.jpg}
              }
              \caption{線上解題服務器架構}
              \label{arc4}            
            \end{subfigure}
            \caption{系統架構設計 (點擊可看大圖)}
          \end{figure}
      
          \par 主要後端服務器(如圖\ref{arc2})
            負責處理與管理使用者資料(User Module)、課程模組(Course Module)、題目模組(Online Judge)。
          
          \par 課程管理服務器(如圖\ref{arc3})
            包括直播服務器(Stream Server)、互動式講義模組(Interactive-book Module)、練習模組(Exercise Module)。
            投影片模組將使用 CodeMirror 作為編輯器,並且提供使用 Markdown 撰寫互動式講義。
            前端呈現如圖\ref{arc5},學生能夠同時看到教師的投影片及操作。
            % 接著使用
          \begin{figure}[H]
            \centering
            \href{https://raw.githubusercontent.com/programingtw/proglearn-plan/main/img/student.png}{ 
              \includegraphics[width=0.8\textwidth]{./img/student.png}
            }
            \caption{學生的前端課堂頁面 (點擊可看大圖)}
            \label{arc5}
          \end{figure}
          \par 線上解題服務器(如圖\ref{arc4})
            包括線上題目抓取模組(Judge API-Client)、程式碼批改模組(Judger)、程式碼分析模組(Code Analysis)。
             Judge API-Client 是使用開源專案 api-client \cite{apiclient} 作為線上題目抓取模組,方便教師可以使用現有的題目做修改。
             Judger 是參考開源專案 go-judge \cite{judger1} 、JudgeServer \cite{judger2} 開發出沙盒程式碼執行環境及程式碼批改模組。
             Code Analysis 將採用多標籤辨識模型實現,後續章節將針對此核心技術進行更詳細的說明。
        
          \item 系統介面設計
            \par 系統的使用頁面主要為課堂教學頁面。並針對不同的使用者,設計兩種使用介面:
            \begin{itemize}
              \item 教師:能夠管理投影區與互動區,並取得學生的答題狀況。
              \item 學生:能夠觀看投影片、互動區的內容,互動區包含滾拉式的 Markdown 講義,並且嵌入以程式為主的互動性問題,讓學生作答。
            \end{itemize}
          \end{enumerate}  
    \end{enumerate}
  \item 開發技術介紹
    \begin{enumerate}[label=(\arabic*)]
      \setlength{\parindent}{2em}
      \item 程式語言:TypeScript, Golang
      \item 前端設計:Vue3, Element Plus, chart.js, Codemirror, Vue Router, Pinia   
      \item 後端設計:Gin, Websocket, WebRTC
      \item 資料庫:PostgreSQL
      \item 開發工具:Git, Docker, VSCode, Postman
    \end{enumerate}
  
  \item 作品展示
    \begin{figure}[H]
      \begin{subfigure}{0.5\linewidth}
        \centering
        \href{https://raw.githubusercontent.com/programingtw/proglearn-plan/main/img/list.png}{ 
          \includegraphics[width=1\textwidth]{./img/list.png}
        }
        \caption{課程清單頁面}
      \end{subfigure}
      \label{arc6}
      \begin{subfigure}{0.5\linewidth}
        \centering
        \href{https://raw.githubusercontent.com/programingtw/proglearn-plan/main/img/course.png}{ 
          \includegraphics[width=1\textwidth]{./img/course.png}
        }
        \caption{課程資訊頁面}
      \end{subfigure}
      \label{arc7}
      \caption{課程相關頁面 (點擊可看大圖)}
    \end{figure}

    \begin{figure}[H]
      \begin{subfigure}{0.5\linewidth}
        \centering
        \href{https://raw.githubusercontent.com/programingtw/proglearn-plan/main/img/student.png}{ 
          \includegraphics[width=1\textwidth]{./img/student.png}
        }
        \caption{學生課堂頁面}
        \label{arc15}
      \end{subfigure}
      \begin{subfigure}{0.5\linewidth}
          \centering
          \href{https://raw.githubusercontent.com/programingtw/proglearn-plan/main/img/teacher.png}{ 
            \includegraphics[width=1\textwidth]{./img/teacher.png}
          }
        \caption{教師課堂頁面}
        \label{arc9}
      \end{subfigure}
      \caption{課堂介面 (點擊可看大圖)}
    \end{figure}
    
    \begin{figure}[H]
      \centering
      \href{https://raw.githubusercontent.com/programingtw/proglearn-plan/main/img/feedback.png}{ 
        \includegraphics[width=0.55\textwidth]{./img/feedback.png}
      }
      \caption{課後作業系統的教師回饋頁面 (點擊可看大圖)}
      \label{arc14}
    \end{figure}
    
    \begin{figure}[htb]
      \centering
      \begin{subfigure}{0.45\linewidth}
        \centering
        \href{https://raw.githubusercontent.com/programingtw/proglearn-plan/main/img/testcode.png}{
          \includegraphics[width=1\textwidth]{./img/testcode.png}
        }
        \caption{題目}
        \label{arc10}            
      \end{subfigure}
      \begin{subfigure}{0.45\linewidth}
        \centering
        \href{https://raw.githubusercontent.com/programingtw/proglearn-plan/main/img/hw_ac.png}{
          \includegraphics[width=1\textwidth]{./img/hw_ac.png}
        }
        \caption{ac}
        \label{arc11}
      \end{subfigure}
      \bigskip
      \begin{subfigure}{0.45\linewidth}
        \centering
        \href{https://raw.githubusercontent.com/programingtw/proglearn-plan/main/img/hw_wa.png}{
          \includegraphics[width=1\textwidth]{./img/hw_wa.png}
        }
        \caption{wa}
        \label{arc12}
      \end{subfigure}
      \begin{subfigure}{0.45\linewidth}
        \centering
        \href{https://raw.githubusercontent.com/programingtw/proglearn-plan/main/img/hw_re.png}{
          \includegraphics[width=1\textwidth]{./img/hw_re.png}
        }
        \caption{re}
        \label{arc13}            
      \end{subfigure}
      \caption{課後作業系統的學生頁面 (點擊可看大圖)}
    \end{figure}

  \item 未來與展望
    \par 未來將增加與改善以下四種功能:
    \begin{enumerate}
      \item 增加自動回饋系統,讓教師可以快速了解學生的學習進度和弱點,減少教師在教學與備課的工作量。
      \item 增加智能助教系統,自動批改學生作業並提出建議,讓教師能夠快速掌握學生的學習狀況。
      \item 增加互動式講義的功能,讓教師與學生之間有更多互動,將抽象的概念視覺化,讓學生更容易理解並加深印象。
      \end{enumerate}
    \par 並且以創業與經營為目標,提供教學系統服務給程式教育的教師,並以全台灣的高中老師作為主要推廣目標。

  \item 參考文獻
    \renewcommand{\section}[2]{}
    \begin{thebibliography}{99}
      \bibitem{ref2} 十二年國民基本教育課程綱要。民112年2月14日,取自:https://www.naer.edu.tw/PageSyllabus?fid=52。
      \bibitem{ref3} 政府資料公開平台(民111年6月29日)。全臺灣各級學校之學生數及畢業生數資料。民112年2月14日,取自:https://data.gov.tw/dataset/31436。
      %\bibitem{ref5} 聯合新聞網(民110年3月8日)。中小學資訊教師荒! 多校找自然師兼任遭批不專業。民112年2月14日,取自:https://udn.com/news/story/6885/5303319。
      \bibitem{ref4} 張瑞賓、李建華。"遠距教學常態化問題之探討與建議。" 臺灣教育評論月刊 10.6 (2021): 27-34。
      \bibitem{ref7} 岳修平、梁朝雲。 "綜整學生,教師與教學情境考量的遠距教學預測模型。" 教育資料與圖書館學 52.1 (2015): 33-57+。
      %\bibitem{ref8} De Giusti, Armando. "Book review: Policy brief: Education during COVID-19 and beyond." Revista Iberoamericana de Tecnología En Educación y Educación En Tecnología 26 (2020): 110-111.
      %\bibitem{ref9} Tumwesige, Josephine. "COVID-19 Educational disruption and response: Rethinking e-Learning in Uganda." University of Cambridge (2020).
      %\bibitem{ref10} Santos, Joseline M., and Rowell DR Castro. "Technological Pedagogical content knowledge (TPACK) in action: Application of learning in the classroom by pre-service teachers (PST)." Social Sciences \& Humanities Open 3.1 (2021): 100110.
      %\bibitem{ref11} Ratheeswari, K. "Information communication technology in education." Journal of Applied and Advanced research 3.1 (2018): 45-47.
      %\bibitem{ref13} Ouya, Samuel, et al. "WebRTC platform proposition as a support to the educational system of universities in a limited Internet connection context." 2015 5th World Congress on Information and Communication Technologies (WICT). IEEE, 2015.
      %\bibitem{ref14} Ibrahim, Mohamed, and Osama Al-Shara. "Impact of interactive learning on knowledge retention." Lecture Notes in Computer Science 4558 (2007): 347.
      %\bibitem{ref15} Akram, Huma, et al. "Technology integration in higher education during COVID-19: An assessment of online teaching competencies through technological pedagogical content knowledge model." Frontiers in psychology 12 (2021): 736522.
      %\bibitem{ref16} Krusche, Stephan, and Andreas Seitz. "Artemis: An automatic assessment management system for interactive learning." Proceedings of the 49th ACM technical symposium on computer science education. 2018.
      %\bibitem{ref17} Dong, Yu, Jingyang Hou, and Xuesong Lu. "An intelligent online judge system for programming training." Database Systems for Advanced Applications: 25th International Conference, DASFAA 2020, Jeju, South Korea, September 24–27, 2020, Proceedings, Part III 25. Springer International Publishing, 2020.
      %\bibitem{ref18} Delen, Erhan, Jeffrey Liew, and Victor Willson. "Effects of interactivity and instructional scaffolding on learning: Self-regulation in online video-based environments." Computers \& Education 78 (2014): 312-320.
      \bibitem{apiclient} api-client. Available from: https://github.com/online-judge-tools/api-client
      \bibitem{judger1} go-judge. Available from: https://github.com/criyle/go-judge
      \bibitem{judger2} JudgeServer. Available from: https://github.com/helsonxiao/JudgeServer
    \end{thebibliography} 
\end{enumerate}
\end{document}